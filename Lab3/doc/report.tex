\documentclass[12pt]{article}

\usepackage{fullpage}
\usepackage{multicol,multirow}
\usepackage{tabularx}
\usepackage{listings}
\usepackage{pgfplots}
\usepackage[utf8]{inputenc}
\usepackage[russian]{babel}

% Оригиналный шаблон: http://k806.ru/dalabs/da-report-template-2012.tex

\begin{document}

\section*{Лабораторная работа №3\, по курсу дискрeтного анализа: Исследование качества программ}

Выполнил студент группы М8О-212Б-22 МАИ \textit{Юрков Евгений}.

\subsection*{Условие}


Для реализации словаря из предыдущей лабораторной работы, необходимо провести исследование скорости выполнения и потребления оперативной памяти.
В случае выявления ошибок или явных недочётов, требуется их исправить.

Результатом лабораторной работы является отчёт, состоящий из:

\begin{itemize}
    \item Дневника выполнения работы, в котором отражено что и когда делалось, какие средства использовались и какие результаты были достигнуты на каждом шаге выполнения
    лабораторной работы.
    \item Выводов о найденных недочётах.
    \item Сравнение работы исправленной программы с предыдущей версией.
    \item Общих выводов о выполнении лабораторной работы, полученном опыте.
\end{itemize}

Минимальный набор используемых средств должен содержать утилиту gprof и библиотеку dmalloc,
однако их можно заменять на любые другие аналогичные или более развитые утилиты (например, Valgrind или Shark)
или добавлять к ним новые (например, gcov).

\newpage
\subsection*{Метод решения}

В рамках выполнения лабораторной работы я буду использовать следующие утилиты:
\begin{itemize}
    \item Анализ времени работы: gprof
    \item Анализ потребления памяти: valgrind
\end{itemize}

% \newpage
\subsection*{gprof}

Утилита gprof позволяет измерить время работы всех функций в программе, 
а также количество их вызовов и долю от общего времени работы программы в процентах.

Для работы с утилитой gprof было необходимо скомпилировать программу с ключом \texttt{-pg}
После запуска полученного исполняемого файла появился файл gmon.out, в котом содержалась информация
предоставленная для анализа программы. Далее этот файл был обработан gprof для получения текстового
файла с подробной информацией о времени работы и вызовах всех функций и операторов,
которые использовались в программе.

\begin{tabular}{ | c | c | c | c | }
    \hline
        \% time & self seconds & calls & name \\
        \hline 
        33.35 & 0.01 & 3448130 & \texttt{map<...>::\_\_data::operator!=(...)} \\
        33.35 & 0.01 & 3397716 & \texttt{bool std::operator< (...)} \\
        33.35 & 0.01 & 75095 & \texttt{rb\_tree<...>::\_\_find\_parent(...)}\\
    \hline
\end{tabular}

Время работы остальных функций по данным gprof заняло меньше 0.01 секунды, поэтому они не были внесены в таблицу.
Как можно заметить из таблицы, больше всего времени заняли функции сравнения ключей и поиска в дереве.
Это связано с тем, что сравнение строк, являющихся ключами, происходит за линейное время, и функция
сравнения вызывается на каждой итерации при поиске в дереве.

\subsection*{valgrind}

Valgrind является утилитой для поиска ошибок в работе с памятью в программе, таких
как утечки памяти и выход за границу массива.

В результате исследования программы valgrind была выявлена утечка памяти в функции очистки дерева. Для её устранения был написан деструктор для структуры \texttt{\_\_node}, 
удаляющий также правого и левого потомков.

Все утечки были устранены. Результат работы valgrind: \\
\texttt{
    ==2555== Memcheck, a memory error detector \\
    ==2555== Copyright (C) 2002-2017, and GNU GPL'd, by Julian Seward et al. \\
    ==2555== Using Valgrind-3.15.0 and LibVEX; rerun with -h for copyright info \\
    ==2555== Command: ./build/Lab2/lab2 \\
    ==2555==  \\
    ==2555==  \\
    ==2555== HEAP SUMMARY: \\
    ==2555==     in use at exit: 122,880 bytes in 6 blocks \\
    ==2555==   total heap usage: 25,073 allocs, 25,067 frees, 2,000,336 bytes allocated \\
    ==2555==  \\
    ==2555== LEAK SUMMARY: \\
    ==2555==    definitely lost: 0 bytes in 0 blocks \\
    ==2555==    indirectly lost: 0 bytes in 0 blocks \\
    ==2555==      possibly lost: 0 bytes in 0 blocks \\
    ==2555==    still reachable: 122,880 bytes in 6 blocks \\
    ==2555==         suppressed: 0 bytes in 0 blocks \\
    ==2555== Reachable blocks (those to which a pointer was found) are not shown. \\
    ==2555== To see them, rerun with: --leak-check=full --show-leak-kinds=all \\
    ==2555==  \\
    ==2555== For lists of detected and suppressed errors, rerun with: -s \\
    ==2555== ERROR SUMMARY: 0 errors from 0 contexts (suppressed: 0 from 0) \\
}

Сообщение \texttt{still reachable: 122,880 bytes in 6 blocks} происходит из-за использования \texttt{ios::sync\_with\_stdio()}, это не является утечкой памяти.

\newpage
\subsection*{Выводы}

В ходе выполнения лабораторной работы я научился выявлять узкие места и проблемы в работе с памятью.
Анализ времени работы программы с помощью утилиты gprof позволяет определить, какие функции занимают больше всего времени,
что помогает в оптимизации кода для улучшения производительности. Анализ работы с памятью с помощью valgrind помогает выявить утечки памяти,
неправильное использование указателей и другие проблемы, которые могут привести к непредсказуемому поведению программы.

Исследование качества программ необходимо для создания надежного, эффективного и безопасного программного обеспечения.
Это позволяет улучшить пользовательский опыт, уменьшить количество ошибок и сбоев, повысить производительность и снизить затраты на поддержку.

\end{document}
