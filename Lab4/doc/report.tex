\documentclass[12pt]{article}

\usepackage{fullpage}
\usepackage{multicol,multirow}
\usepackage{tabularx}
\usepackage{listings}
\usepackage{pgfplots}
\usepackage[utf8]{inputenc}
\usepackage[russian]{babel}
\usepackage{pgfplots}
\usepackage{tikz}

% Оригиналный шаблон: http://k806.ru/dalabs/da-report-template-2012.tex

\begin{document}

\section*{Лабораторная работа №4\, по курсу дискрeтного анализа: Поиск образца в строке}

Выполнил студент группы М8О-212Б-22 МАИ \textit{Юрков Евгений}.

\subsection*{Условие}

\textbf{Вариант:} 2-1

Необходимо реализовать один из стандартных алгоритмов поиска образцов для указанного алфавита.

\textbf{Вариант алгоритма:} Поиск одного образца при помощи алгоритма Бойера-Мура.

\textbf{Вариант алфавита:} Слова не более 16 знаков латинского алфавита (регистронезависимые).


\newpage
\subsection*{Метод решения}

Для написания алгоритма Бойера-Мура сначала необходимо осуществить предподсчёт для искомого образца: составить таблицу плохих символов
и массив хороших суффиксов. Сравнение образца с подстрокой осуществляется справа налево. На каждом этапе сдвиг осуществляется на
$max(1, ППС_i, ПХС_i)$, где ППС - правило плохого символа, ПХС - правило хорошего суффикса.

% \newpage
\subsection*{Описание программы}

Для реализации алгоритма были реализованы следующие функции:
\begin{itemize}
    \item \texttt{Z} - z-функция, используется для составления массива хороших суффиксов;
    \item \texttt{bad\_symbol} - построение талблицы плохих символов;
    \item \texttt{good\_suf} - построение массива хороших суффиксов;
    \item \texttt{binsearch} - используется для поиска нужного индекса в таблице плохих символов;
    \item \texttt{get\_offset} - функции для получения значения из правил;
    \item \texttt{boyer\_moore} - главная функция для поиска образца в строке;
\end{itemize}

\newpage
\subsection*{Дневник отладки}

\begin{enumerate}
    \item исправлена нумерация строк
    \item исправлена неккоректная работа поиска плохого символа
    \item вектор с нумерацией слов исправлен на хеш-таблицу
\end{enumerate}

\newpage
\subsection*{Тест производительности}

Алгоритм Бойера-Мура для поиска образца в тексте работает за "сублинейное" время, то есть сложность $O(n+m)$, где n - длина текста.

\begin{tikzpicture}
\begin{axis}[xlabel={Время, ns}, ylabel={Длина текста, строки}]
\addplot coordinates {
    (662149, 500)
    (14323680, 1000)
    (53270937, 5000)
    (3188399406, 50000)
    (31473939674, 500000)
};
\end{axis}
\end{tikzpicture}


\newpage
\subsection*{Выводы}


Я создал функцию для поиска слова в тексте, используя алгоритм Бойера-Мура. 
Однако, в некоторых сценариях данная функция может немного замедляться,
 особенно если в тексте много совпадений или много коротких слов. Несмотря на это, алгоритм Бойера-Мура остается одним из самых эффективных
 алгоритмов поиска шаблонов в стандартных приложениях и командах, таких как поиск по Ctrl+F в браузерах и текстовых редакторах.
\end{document}
